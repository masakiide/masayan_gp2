\documentclass[../main.tex]{subfiles}
\begin{document}
\chapter{序論}

本章では本研究の背景、課題及び手法を提示し、本研究の概要を示す。


\section{動機}
水球とは,縦30m横25mのプールを使用して行われる球技で試合時間は8分\times4ピリオドである.
各チーム,ゴールキーパー1名ずつを含む計7名の選手がフィールド上におり,1度のオフェンスを30秒以内に行うことがルールとされている.
2019年のFINAによるルール変更以来,試合のスピード感を上げ得点機会を多く作るために審判がパーソナルファウル
の判断を下すことが増えており,オフェンス・ディフェンスの両側面で退水セットの重要性が増している.
2019年に行われた世界選手権ではルール改正の影響より退水セットのゴール数,シュート数が増加している.\cite{weko_1677_1}
2021年に行われた東京オリンピック水球競技での決勝ラウンドでは,退水セットの決定率が対戦相手より低いチームは全て敗退するなど\cite{weko_1697_1},
退水セットの重要性が伺える大会であった.
\section{目的}
さて、慶應義塾體育會水泳部水球部門\{以下慶應水球\}は2022年度関東水球学生リーグ4にて位入賞という47年ぶりの快挙を成し遂げたが,
上位3チームとの差は大きく様々な面でのチーム強化が求められることとなった.
その中で執筆者が着目したのは退水セットである.2022年度学生リーグ1部で行われた33試合中,パーソナルファアウルの回数は計473回であり,
1試合平均14.3回のパーソナルファウルが起きているという計算になる.
この様に退水セットの重要性は増しつつあるが,現状の慶應水球では退水セットの練習について決め事をミーティングで決めた後に
練習を行い,各自が帰宅後にビデオを視聴するという主観的な形態の分析にとどまっている.
日本の水球競技は、各チームの指導者が長年の経験と勘による指導を行うケースが非常に多い.\cite{原朗2005水球競技の長期一貫指導型競技者育成プログラム}
そのため選手も経験と勘・癖を頼りにプレーを行っていることが多く,国際大会で上位入賞するチームとは戦術が大きな乖離があると言われている.\cite{榎本至2005水球競技のノーティカルチャート}
過去の研究では日本代表がサイドからのシュートが多いが欧州の強豪国は中からの得点率が高く,よりゴールの可能性が高い状況に持ち込むことが出来ていることが分かっている.\cite{高木英樹1988092o09}\cite{洲雅明2018水球競技における相手退水時の攻撃分析}
そのため退水セットの分析ツールを作成し対戦相手の分析を行うことで試合を有利に進めることができると考えた.
退水セットのおける各ポジションからの一般的な平均パスコースの分析を行った研究はあるものの\cite{洲雅明2016水球競技における退水時攻撃のディフェンスの崩しについて},
エクセルを用いたものであり日常の練習から利用できる分析体系ではないと考えた.
他にも,試合中の選手の動きについてVTR映像を基にVTRモーションアナライザーやビデオ分析ソフトから座標値を求め,2次元DLT法で泳速度や泳距離,コースを分析した研究が存在する.\cite{椿本1987水球のゲーム分析}\cite{清水信貴2007水球競技におけるルール改正に伴うゲーム構造の変化に関する研究}\cite{丸山博史水球競技における選手及びボールの移動からみたチームパフォーマンス}\cite{椿本昇三19839043}\cite{福中賢一女子水球選手の国際試合における試合分析}\cite{椿本昇三1987091219}
しかしこれらを実践に活用したと言う報告はない上,迅速な分析が可能ではないと考えた.
また,水球の試合データを扱った研究についても多数あるが,どれも統計情報のデータベースを作成して試合の傾向を分析することに終始している.\cite{洲雅明2013ロンドンオリンピックにおける水球競技のデータ分析}\cite{洲雅明2021水球男子日本チームの世界選手権}\cite{宮城進1985水球競技におけるゲーム分析}\cite{weko_1697_1}\cite{南隆尚1995水球競技におけるゲーム分析}\cite{宮城進1985093231}\cite{榎本至1998095c03102}\cite{高山誠1985093230}
そこで本研究では退水セットのボール回しを簡易的に可視化し,統計情報やビデオ視聴に加えた分析を行うことで
退水セットの練習において従来以上の効果をもたらすことができると考えた.また,チーム毎に設定している退水セットの攻め方を分析することで
対戦相手に対して有利な状況で試合を進めることができるようになると考えた.
簡易的に操作ができるかつ選手のプレーに影響を与えることのできる
そこで本研究では退水セットのボール回しを可視化するツールを開発し,練習後の分析にツールを使用したチームと
使用しなかったチームでは2回行われた退水セットの練習でどのような変化が見られるのかを計測した.
特に2度の退水セットにおけるパス数やシュート数などの統計情報に加え,出力結果からどのような分析内容が得られたのかにも着目し,実用を意識した検証を行った.


\end{document}
