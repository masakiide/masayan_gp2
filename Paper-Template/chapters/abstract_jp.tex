\documentclass[../main.tex]{subfiles}
\begin{document}

概要
\begin{center}
    \begin{large}
        \begin{tabular}{cc}
            \hline
                \titlejp \\
            \hline
        \end{tabular}
    \end{large}
\end{center}



\par 水球とは,縦30m横25mのプールを使用して行われる球技で試合時間は8分\times4ピリオドである.
各チーム,ゴールキーパー1名ずつを含む計7名の選手がフィールド上におり,1度のオフェンスを30秒以内に行うことがルールとされている.
2019年のFINAによるルール改正以来,水球界においては得点機会を増やすために審判が退水やペナルティーファールの判断を下すことが増えており,
オフェンス・ディフェンスの両側面で退水セットの重要性が増している.
2021年に行われた東京オリンピック水球競技での決勝ラウンドでは,退水セットの決定率が対戦相手より低いチームは全て敗退するなど退水セットの重要性が伺える大会であった.
このように試合の勝敗を左右する退水セットであるが,現在の慶應義塾大学體育會水泳部水球部門では退水セットの練習では,シュート数やパスミスなどの統計情報と
ビデオを各自視聴をするだけの分析に止まり,主観的な分析にとどまっている.
\par そこで本研究ではPythonに付属の標準ライブラリであるtkinterを用いて退水セットのボール回しを可視化するツールを開発した.
これを用いて統計情報やビデオ視聴に加えた客観的な分析を行うことで退水セットの練習において従来以上の効果をもたらすことを主題とした.
練習後の分析にツールを使用したチームと使用しなかったチームでは2回行われた退水セットの練習で
どのような変化がOFとDFに見られるのかをそれぞれ計測した.
その結果,DFへの適応実験ではツール不使用チームが相手OFへの対策を立てられずに2度の退水セットで同じ失点の仕方を繰り返していたのに対し,ツール使用チームは
相手の攻め方を見抜き,対策を立てることで同じ失点の仕方をすることがなくなった.アンケートから客観的な分析による心理変化があったことも分かっており,選手の行動に
影響を与える分析結果を与えることが出来た.
OFへの適応実験では,ツール使用チームと不使用チームでは統計上の差が見られなかった.アンケートから心理的な変化があったことは分かったもののプレーに影響を与えることは出来なかった.



キーワード:\\
\underline{1. 卒業論文},
\underline{2. 水球},
\underline{3. 退水セット},
\underline{4. tkinter}

\rightline{\departmentjp}
\rightline{\namejp}

\clearpage

\end{document}
