\documentclass[../main.tex]{subfiles}
\begin{document}
\chapter{結論}

本章では,本研究のまとめと今後の展望を示す.

\section{本研究のまとめ}
本研究では,選手の動きに影響を与えることのできる,有用性がありかつ簡易的に使用できるツールの作成を主題とした.
まず選手の動きへの影響については,
\par DFへの適応:ツール使用チームと不使用チームでは,2度の退水セットで相手のOFへの対応に明確な差が現れた.
本研究で提案した手法により客観的な分析が可能になり心理的な変化が
もたらされプレーに影響を与える分析結果を与えることができた.
\par OFへの適応:ツール使用チームと不使用チームでは,2度の退水セットでのボール回しに有意性のある差は見られなかった.
心理的な変化は見られたものの戦術への目に見えた反映は行われなかった.
\par ツールのユーザーインターフェイス:出力結果が動画ではなく画像でも一部プレーに影響を与えられる出力が得られた.操作1〜5ではパスの入力,出力結果の再生において
は設計者とユーザーの操作性に乖離があることが分かった.



\section{本研究の課題}
本研究で得られた課題として,実験期間が短く結果を長期間に渡って測定できなかったことが挙げられる.
定期的に分析結果に基づくフィードバックを与えることでOFにおいても改善が見られた可能性は大きい.
また,作成したツールについてはボールの動きが往復した時、描画の線が重なるタイミングがあり見にくいことや操作が難解な箇所があり,改善が求められる.


\section{今後の展望}
現在の水球界はICT化があまり進んでおらず,試合の公式データは全て手動でとっているという現状がある.大学水球に関しては1名しかいない記録係がいる時にしかデータを取得できず,
記録係不在の学生リーグでの試合は公式データをとることができていない.今後サッカーやアメリカンフットボールをはじめとする他競技のように,試合映像からプレーの種別を判断し,
自動でラベリングできるようにしたい.そうすることで水球の分析が目に見える統計情報だけでなく,目では見えない分野の分析まで進むのではないだろうか.
そのような分析の集積が今後の日本の水球を変え,日本の水球強化につながることを望む.


\end{document}