\documentclass[../main.tex]{subfiles}
\begin{document}
\chapter{評価}

まだちゃんと整理できていません
本実験では、戦術への適応とツールの使用感という2つの側面からの評価を行う.\\
戦術への適応に関しては2回に渡って行われた退水セットの中から,統計上の数値変化を比較する.
それにより,分析結果が戦術に反映できるものであったのかを評価する.\\
ツールの使用感に関してはツール使用後のアンケートや操作時間,操作の正確性について評価を行う.


\section{ディフェンスへの適応実験の評価}

\subsection{統計情報での変化}
スタッツ比較の場所です
\par ここでは,分析対象となったBチームのオフェンスについて,2度の退水セットでどのような変化が現れたのかを統計情報から評価を行う.
まず,1度目と2度目の退水セットでの1番〜9番の各ポジションでのボールタッチ数はこのようになる.\ref{table:青チームのポジション毎のボールタッチ数}\ref{table:緑チームのポジション毎のボールタッチ数}
\par 分析対象となったBチームの変化について,まず着目するべきは5番・6番・7番ポジションでの統計情報の変化となる.1度目の退水セットでは「中、つまり5番〜9番を使う」という
コンセプトの下,右サイドである1番・2番で多くのパス交換を行いDFであるAチームの選手を揺さぶり5・6・7番を効果的に扱い、10本の退水セットの内7度中にパスを通し6度の
シュートを放ち4本のゴールを決めている.これに対し,2度目の退水セットでは対戦相手であるAチームは分析結果に基づいたDFを行った.その結果,2度目の退水セットでは
5番・6番へのパスを3本に減らし,1度もシュートを打たせることなく全てをカットしている.


\par 分析を行ったAチームのOFについてまとめると次のようになる.1度目の退水セットでは10本中で6度のパスを通し,4本のシュートを打ち2本のゴールを決めている.
2度目の退水セットについては5度のパスを中のポジションに出し,3本のシュートを放っている.結果として中からの得点は0得点であったが,2度の退水セットでそれぞれ6本・5本の
パスを中に通すことができており,分析を行っていないBチームはAチームの戦術に対応することができていなかったと言える.
以上のことから,本研究で作成したツールによる分析結果からAチームのDFを改善させ、BチームのOFへの適切な対策を立てることが出来たと考えられる.



\begin{table}[h]
  \caption{Aチームのポジション毎のボールタッチ数}\label{table:青チームのポジション毎のボールタッチ数}
  \begin{tabular}{lllllllllll}
    \hline\hline
  ポジション番号 & 1番 & 2番 & 3番 & 4番 & 5番 & 6番 & 7番 & 8番 & 9番 & 合計 \\
  \hline\hline
  1回目パス      & 22 & 13 & 9 & 7 & 2  & 4  & 0  & 0  & 0  & 45 \\
  1回目シュート      & 1 & 1 & 1 & 0 & 2  & 2  & 0  & 0  & 0  & 7 \\
  1回目ゴール     & 0 & 1 & 0 & 0  & 2  & 0  & 0  & 0  & 0  & 3 \\
  \hline
  2回目パス     & 17 & 11 & 11 & 11  & 2  & 3  & 0  & 0  & 0  & 45 \\
  2回目シュート     & 2 & 1 & 0 & 2  & 0  & 3  & 0  & 0  & 0  & 8 \\
  2回目ゴール     & 1 & 1 & 0 & 1  & 0  & 0  & 0  & 0  & 0  & 3 \\
  \hline
  \end{tabular}
  \end{table}

\begin{table}[h]
  \caption{Bチームのポジション毎のボールタッチ数:Aチームによる分析対象}\label{table:緑チームのポジション毎のボールタッチ数}
  \begin{tabular}{lllllllllll}
    \hline\hline
  ポジション番号 & 1番 & 2番 & 3番 & 4番 & 5番 & 6番 & 7番 & 8番 & 9番 & 合計 \\
  \hline\hline
  1回目パス      & 16 & 15 & 15 & 14 & 2  & 4  & 1  & 0  & 0  & 67 \\
  1回目シュート     & 1 & 0 & 1 & 1  & 3  & 2  & 1  & 0  & 0  & 9 \\
  1回目ゴール     & 1 & 0 & 1 & 0  & 1  & 2  & 1  & 0  & 0  & 6 \\
  \hline
  2回目パス      & 16 & 12 & 13 & 9  & 3  & 0  & 0  & 0  & 0  & 53 \\
  2回目シュート     & 1 & 0 & 4 & 2  & 0  & 0  & 0  & 0  & 0  & 7 \\
  2回目ゴール     & 0 & 0 & 1 & 3  & 0  &  0 & 0  & 0  & 0  & 4 \\
  \hline
  \end{tabular}
  \end{table}



\subsection{ツール使用の前後での心理変化}
\par ここでは,分析を行ったことで2度の退水セットでのプレーに対してどのような心理面の変化があったのか,ツール使用者7名にアンケートをとった.\ref{table:青チームの心理変化}
多くの選手から「中」という言葉が出てくるように,分析の出力結果から相手のコンセプトである「中からの失点」に意識が向いていたことが分かる.
以上のことから,本研究においてツールの使用により選手の意識を変え,
心理面と競技面から選手に影響を与えることが出来たと考えられる.



\begin{table}[h]
  \caption{Aチームの選手の心理変化}\label{table:青チームの心理変化}
  \begin{tabular}{llll}
    \hline\hline
  被験者 & 分析後、& どのような変化がありましたか? & DFの改善に\\
  & 自分達のDFに & & この分析は使えそうか \\
  & 変化を感じたか & & \\
  \hline\hline
  A1     & y & 中を意識するという認識が全員にあったので、& y \\
  & & 前よりも声が出ていて守りやすかった。 &  \\
  A2     & y & 相手の責め方がわかった & y \\
  A3     & y & 強く当たるところがわかった、 & y \\
  & & シュートが多くくるコースがわかった & \\
  A4     & y & 相手のコンセプトを意識してdfするようになった & y \\
  A5     & y & 中での失点が減った & y \\
  A6     & y & DFを意識的に行えた & y \\
  A7     & y & 中を重点的に守った & y \\
  \hline
  \end{tabular}
  \end{table}


\par ここでは,ツールでの出力結果を基にAチームの選手が行った分析6名分を記載する.\ref{table:青の分析内容一つ目}\ref{table:青の分析内容二つ目}
なお,本研究では通常の分析に加えて分析の効果を確かめるため,相手であるBチームのコンセプトを予想し,それを確かめる質問も行った.
1つ目の質問であるBチームのOFにおけるコンセプト予想については,『分からない』が3名・正しく答えられた選手が2名・異なる回答が1名であった.





  \begin{table}[h]
    \caption{Aチームが行った分析1/2}\label{table:青の分析内容一つ目}
    \centering
    \begin{tabular}{lllll}
      \hline \hline
      被験者 & コンセプト予想 & 分析結果から分かること \\
      \hline \hline
      A1 & わからなかった。& 右サイドで多くボールを回していることがわかった。\\
      &&ゴールの左隅に打つ。基本的に中が上がったりはしない。\\
      &&5,6番が空いたら積極的に使う。中は潰して、\\
      &&上はキーパーと連携してハンドアップ。\\
      A2 &シュート力のある選手で勝負&図形にしてみると非常にわかりやすく、1,3と2,4のパスが\\
      &&多く行われている。そこのパスカットをもう少し狙えばよかった。\\
      A3 & 5,6を使う & 5、6フィニッシュが10本中6本あったので、やはり、 \\
      &&中でフィニッシュするのがコンセプトだと思う\\
      A4 & わからない、中? & 右ポストにシュートが飛んでくることが多いので、\\
      &&ハンズアップかキーパーコースか徹底させる。\\
      &&中にボールが行くことが多いので、中のディフェンスを徹底させる。\\
      A5 & わからない & 7回ほどのパス回しが一番多いこと。 \\
      A6 & 分からない & 相手のパス回しの傾向やシュートまでの回数を数値化できるかも\\
      &&しれないと思ったまたゴールとシュート、カットの時とで。\\
      &&何が違うのかを分析することもできると思った。 \\
      \hline
    \end{tabular}
    \end{table}

  \begin{table}[h]
    \caption{Aチームが行った分析2/2}\label{table:青の分析内容二つ目}
    \centering
      \begin{tabular}{lllll}
    被験者 & 分析結果から分からないこと & 相手のコンセプトは何か\\
    \hline\hline
    A1&どこのハンドアップが抜かれたかなど。	&上からあまり打たない。中を積極的に使う。\\
    A2&どのタイミングでどの選手が2メートルを切っているか&ボール保持の時間を多くとらずに\\
    &パスはジャストパスだったのか・キーパーの位置&積極的に外周で回してLの位置にシュート。\\
    &ハンズアップは上がっていたのか&中意識で無理そうだったら上から打つ感じ。\\
    &総じてOFに特化した分析ツールであると感じた。&\\
    A3&センタリングの時に、2メートルをえぐられたのか、&5.6フィニッシュ\\
    &えぐられてないのか&\\
    A4&敵のズレやスライド	&中を使う\\
    A5&シュートの上下とバウンドしたか。&中を使う\\
    A6&ボール保持時間やフェイクなどの動きがわからない。&分からない。\\
    &また、ボールの軌道が重なっているところやパス回し&\\
    &の順番、個人の識別(左利きなど)が難しいと感じた。&\\
    \hline
    \end{tabular}
    \end{table}


\subsection{ツールの使用感}
ここではまず,各被験者並びに設計者が1本目の出力終了後から10本目の出力終了までの時間と,10本の退水セットを出力する際に全てのパスを正しく入力できた本数について触れる.\ref{table:大学生の時間と正確性}\\
本実験ではAチームの選手が分析を行ったため,A1〜A7という番号を被験者に振り、記載している.

% ----- 表:時間と正確性 ------
\begin{table}[h]
\caption{各被験者の測定時間と正確性}\label{table:大学生の時間と正確性}\centering
  \begin{tabular}{lcr}
  \hline \hline
  被験者 & 使用時間 & 正確性 \\
    & s & 10個中の正確さ \\
  \hline \hline
  設計者 & 268  & 10 \\
  A1 & 742 & 7 \\
  A2 & 805 & 8 \\
  A3 & 805 &10 \\
  A4 & 700 & 10 \\
  A5 & 1011 & 9 \\
  A6 & 798 & 8 \\
  A7 & 691 &8 \\
  \hline
  \end{tabular}
\end{table}


ここから設計者と各被験者の差で大きな時間の差があることが分かる.
そこで,捜査中の画面収録映像より操作1〜5での10本分の合計時間について,被験者3名の記録をとった.
その他被験者については誤操作により画面収録が出来ていないことから計測できなかった.
なお,10本目の出力後はスクリーンショットの撮影のみ行い画面切り替えを行っていないため,操作5のみ9本分の合計時間となる.
A.avはA2・A3・A4の平均値を算出したものであり,設計者をNとしている.
A.av/Nを計算することで各操作での設計者とユーザーの操作時間の倍率を計算している.
この倍率が高いほど操作が困難であるということであり,改善が必要な箇所となる.
本実験では操作2のパス入力並びに,操作3の再生ボタンを押すまでの操作に問題があることが分かった.\ref{table:各操作時間}\\


\begin{table}[h]
  \caption{各操作の合計時間}\label{table:各操作時間}\centering
    \begin{tabular}{llllll}
    \hline \hline
    被験者 & 操作1  & 操作2 & 操作3 & 操作4 & 操作5  \\
    \hline \hline
    A2 & 136 & 618 & 80 & 100 & 41\\
    A3 & 119 & 549 & 134 & 101 & 65\\
    A4 & 164 & 419 & 76 & 92 & 53 \\
    A.av & 139.67 & 528.67 & 96.67 & 97.67 & 53.00\\
    N & 65 & 199 & 27 & 53 & 34 \\
    \hline
    A.av/N & 2.15 & 2.66 & 3.58 & 1.84 & 1.56 \\
    \hline
    \end{tabular}
  \end{table}
  










\subsection{ツール使用後の感想}

ここでは最後に,ツールを用いて分析を行った選手による,操作終了後の感想やツール使用中の分析者の呟きをまとめたものを記載する.\ref{table:青チームのツール使用者の感想}
本研究で作成したツールはテンキー入力形式ではなく数値を直接入力する形式だったため,数値の誤入力やマウスの移動など,パス数値入力作業における意見が目立った.
これは上述の各操作1〜5における各画面での所要時間とも一致しており,本ツールにおいて最も煩雑な操作となった.

\begin{table}[h]
  \caption{Aチーム選手によるツール使用後の感想}\label{table:青チームのツール使用者の感想}\centering
    \begin{tabular}{lcr}
    \hline \hline
    被験者 & 感想 \\
    \hline \hline
    A1 & 改行の時にマウスで合わせたり移動キーを使うのが少し大変だった。\\
    &タイムアウト時などの退水セットやセットプレーのボールの動きの確認に使えそう。 \\
    & あとは相手のパスカット。 \\
    A2 & 他の部分を消しそうで怖かった。マウスと移動キーを使うのは大変だったけど、 \\
    & だんだん慣れてきた。 \\
    & 作業の効率もどんどん慣れて上がっていくのを感じたけど、 \\
    & 毎回スクショするのは大変だった。 \\
    A3 & 作業自体はあまり大変ではなかった。プログラミングなので \\
    & しょうがないのかもしれないが、\\
    &sかg、打ったところをGoogle Formのように選択できるようにしたら \\
    & もう少し楽かもしれない。\\
    & 後半になるにつれて慣れてきてスピードも上がっていった。 \\
    & ボール回しを視覚化できるのは分析に役立ちそうだと思った。 \\
    A4 & パスがつながる時、1→2、2→3と2は共通しているのでそこの部分をもう一度打つのは \\
    & 手間がかかると思った。\\
    & ボールがどのように動いたかはわかるが、その時にジャストでパスが投げられていたのか \\
    & どうかや、相手の状況が鮮明にわからないので、同じ番号でパスが動いていたとしても \\
    & 一つのデータとして同じにまとめてしまうことに少し疑問を感じた。 \\
    &(2つ全く同じデータになったとしても片方は苦し紛れかもしれないが、それらの判断がつかない)\\
    A5 & 単純な作業であったため作業は簡単だった。\\
    &最後に表示される出力されるものとあっているか確認するのが難しかった。\\
    A6 & 単純作業なので難しくはないが手間がかかる。ボールの動きが重なっているところは \\
    & あっているのか確認がしずらかった。\\
    &入力した数字の通りに結果が出てくるのは面白かった。 \\
    A7 & 単純作業なので作業効率がだんだん上がってくる。サウスポーとか相手のキーマンに \\
    & どう対応するかが難点だと感じた。カーソルを毎回あわせるのが面倒だった。\\
    \hline
    \end{tabular}
  \end{table}
  






\section{オフェンスへの適応実験の評価}

\subsection{統計情報での変化}
スタッツ比較の場所です
\par ここでは,分析対象となったCチームのオフェンスについて,2度の退水セットでどのような変化が現れたのかを統計情報から評価を行う.
まず,1度目と2度目の退水セットでの1番〜9番の各ポジションでのボールタッチ数はこのようになる.\ref{table:Cチームのポジション毎のボールタッチ数}\\



\begin{table}[h]
  \caption{Cチームのポジション毎のボールタッチ数}\label{table:Cチームのポジション毎のボールタッチ数}
  \begin{tabular}{llllllll}
    \hline\hline
    ポジション番号 & 1番 & 2番 & 3番 & 4番 & 5番 & 6番 & 合計 \\
  \hline\hline
  1回目パス      & 18 & 17 & 14 & 12 & 1  & 1  &  63 \\
  1回目シュート     & 0 & 2 & 3 & 3  & 0  & 0  &  8 \\
  1回目ゴール     & 0 & 0 & 2 & 1  & 0  & 0  &  3 \\
  \hline
  2回目パス      & 16 & 15 & 10 & 16  & 0  & 1  &  58 \\
  2回目シュート     & 0 & 2 & 1 & 5  & 1  & 0  &  9 \\
  2回目ゴール     & 0 & 1 & 0 & 3  & 0  &  0 &  4 \\
  \hline
  \end{tabular}
  \end{table}




  \begin{table}[h]
    \caption{Dチームのポジション毎のボールタッチ数}\label{table:Dチームのポジション毎のボールタッチ数}
    \begin{tabular}{llllllll}
      \hline\hline
      ポジション番号 & 1番 & 2番 & 3番 & 4番 & 5番 & 6番 & 合計 \\
    \hline\hline
    1回目パス      & 14 & 11 & 11 & 9 & 1  & 1  &  47 \\
    1回目シュート     & 3 & 1 & 1 & 3  & 1  & 0  &  9 \\
    1回目ゴール     & 1 & 0 & 1 & 0  & 1  & 2  &  6 \\
    \hline
    2回目パス      & 10 & 12 & 10 & 12  & 0  & 3  &  47 \\
    2回目シュート     & 0 & 0 & 4 & 3  & 0  & 3  &  10 \\
    2回目ゴール     & 0 & 0 & 3 & 1  & 0  &  1 &  5 \\
    \hline
    \end{tabular}
    \end{table}


    \subsection{ツール使用の前後での心理変化}
    \par ここでは,分析を行ったことで2度の退水セットでのプレーに対してどのような心理面の変化があったのか,ツール使用者6名にアンケートをとった.\ref{table:高校生チームの心理変化}
    本アンケートについて,C6の被験者はゴールキーパーというポジションのためオフェンスを実施していないためアンケートは実施しなかった.また,C7の被験者は
    2度目の退水セットに参加できなかったことからこちらについてもアンケートを実施しなかった.
    
    
    \begin{table}[h]
      \caption{Cチームの選手の心理変化}\label{table:高校生チームの心理変化}
      \begin{tabular}{lllll}
        \hline\hline
      被験者 & 1度目と & どのような変化がありましたか? & 日々の練習で & 退水以外でどのような\\
      & 2度目で & & この分析は & 場面で分析が使えそうか \\
      & 変化を & &有効そうか& \\
      &感じたか&&&\\
      \hline\hline
      C1 & y & パスは通ってることがちゃんと& y & カウンターでも使いように \\
      &&分かったから、①よりもシュートに&&よっては上手くいきそう\\
      &&意識を割くことが出来た。&&だなと感じた。\\
      C2 & y & 結局上手く行かなかったが、 & y & カウンター時のパス交換など \\
      &&中を見ることを意識することはできた。&&\\
      C3 & y & 前回できなかった動きを& y & カウンターの形の確認や \\
      &&意識することができた &&セットプレーでの有効性\\
      &&&&があると感じた。\\
      C4 & y & 成功パターンを思い出せた。 & y & セットでの攻撃 \\
      &&&&カウンターや\\
      &&&&1対1以外での攻撃\\
      C5 & y & 4番の位置ではパスが下手すぎたので& y & セットオフェンスの時に\\
      &&ダメだったが、5番の位置では&&どの攻め方が1番決まるか\\
      &&上がる場面と中で受ける場面が&&分かる気がしました。 \\
      &&理解できるようになって、 &\\
      &&ゴールに関われるようになった。&\\
      C6 & & & \\
      C7 & & & & \\
      \hline
      \end{tabular}
      \end{table}




ここでは,ツールでの出力結果を基にCチームの選手が行った分析7名分を記載する.\ref{table:高校生の分析内容一つ目}\ref{table:高校生の分析内容二つ目}


\begin{table}[h]
  \caption{Cチームが行った分析1/2}\label{table:高校生の分析内容一つ目}
  \centering
  \begin{tabular}{lllll}
    \hline \hline
    被験者 & 決め事の中で出来たこと & 分析シートを見て振り返り	 \\
    \hline \hline
    C1&自分は主に4番の位置にいて、&分析シートで見ても、①にも述べたように\\
    &今回の決め事で4番はシュートまたは& 4番から2,3,5,6番へのパスはなく、\\
    &1番へのパスの2択であったのもあるが、&シュートか1番へのパスだけだったため\\
    &その2つは徹底してできたと思う。&手応え同様うまくいっていたのだと分かった\\
    &&また、1番へのパスは綺麗に通せた。\\
    C2&それぞれのポジションで、&ほとんどが外周からのシュートで終わっていて、\\
    &パスを決められた通りに&その決定率が低いということも分かる。\\
    &出せている点。全体的に動いて&また、1つ隣へのパスが多いにも関わらず\\
    &プレイをできていた点。&カットが多い。\\
    C3&1番からもらってのシュートを&2番から蹴り上がってもらう\\
    &狙うことができた。&動きができておらず\\
    &&ボールが来ていなかった。\\
    C4&3番から狙いつつ5番に&大体の流れが見えた。\\
    &パスしてシュート&\\
    C5&3番がずれて、かつ4番が&外からのシュートが多め。\\
    &2m付近にいて上がってないときは、&中を使えるときは\\
    &しっかり3.5番みたいな位置に&もっと使ってもいいと思った。\\
    &切りあがって受けてから&\\
    &シュートを狙った&\\
    C6&同サイドハンズアップする&なし\\
    C7&一番が持ったら右にズレる&2番からのシュートが少ない\\
    \hline
  \end{tabular}
  \end{table}

\begin{table}[h]
  \caption{Cチームが行った分析2/2}\label{table:高校生の分析内容二つ目}
  \centering
  \begin{tabular}{lllll}
    \hline \hline
    被験者 & 改善点	 & ここからは不明瞭な部分	 \\
    \hline \hline
    C1&2種類の選択はうまくいっていると&この分析シートの図だけでは、\\
    &分かったので、次に打つ時は&1→4のパスの後の4→1のようなパスの場合、\\
    &きちんと決めきれるようにしたい。&線が重なってしまい分かりづらいかなと感じた。\\
    C2&中へのパスはカットされることが&1人1人がどういうふうに動いていたのかが\\
    &多いのは分かるが、そこも丁寧にするべき。&分からないため、大体の選手の位置や\\
    &また、外周からのシュートは、&ボールの動きのみしか分からない点。\\
    &動いてハンズアップからズレてワンタッチと&\\
    &いうのをもっと狙えたら良いと思う。&\\
    C3&2番からもらうための動きを取り入れて&どのディフェンスにカットされたのか\\
    &1番からもらった時のシュートが&どこのスペースが空いていて、\\
    &入るように動くこと&どこに動くべきだったのかが判断しづらい\\
    C4&もっと自分で狙っていきたい。&20秒のうち何秒使ったのか。\\
    C5&僕が4番の位置をしていた時に&文字があるので分かりますが、図だけだと\\
    &4番から1番へのサイドチェンジの際に&2から3にパスを出してまた2に戻したときなどに、\\
    &パスずれが多くあった。しっかり丁寧に&線がかぶって少し見づらいなと感じました。\\
    &パスをつなげたい。&\\
    C6&なし&なし\\
    C7&気持ちとしてはもっと狙いに行く&中が支えていないから、タイミングを考える\\
      \hline
    \end{tabular}
    \end{table}

\subsection{使用者による操作感}

ここでは,各被験者並びに設計者が1本目の出力終了後から10本目の出力終了までの時間と,10本の退水セットを出力する際に全てのパスを正しく入力できた本数について触れる.
本実験ではCチームの選手が分析を行ったため,C1〜C7という番号を被験者に振り、記載している.\ref{table:高校生の時間と正確性}

% ----- 表:時間と正確性 ------
\begin{table}[h]
    \caption{各被験者の測定時間と正確性}\label{table:高校生の時間と正確性}
    \centering
    \begin{tabular}{clll}
      \hline \hline
      被験者 & 使用時間 & 正確性 \\
      & s & 10個中の正確さ \\
      \hline \hline
      設計者 &  379  & 10 \\
      C1 & 934 & 9 \\
      C2 & 592 & 8 \\
      C3 & 806 & 9 \\
      C4 & 669 & 10 \\
      C5 & 1378 & 10 \\
      C6 & 965 & 9 \\
      C7 & 768 & 10 \\
      \hline
    \end{tabular}
  \end{table}


OFへの適応実験についても,ユーザーである被験者と設計者の操作時間の細かい差を調べるためにDF適応実験と同じ手段を用いた.
C.av/Nの値はこのようになった.\ref{table:各操作時間高校生}
こちらにおいては,倍率がDF適応実験ほど高くないものの操作2と操作3で,ユーザーと設計者の間に乖離があることが分かる.\ref{ta}

\begin{table}[h]
  \caption{各操作の合計時間}\label{table:各操作時間高校生}\centering
    \begin{tabular}{llllll}
    \hline \hline
    被験者 & 操作1  & 操作2 & 操作3 & 操作4 & 操作5  \\
    \hline \hline
    C2 & 117 & 355 & 42 & 85 & 28\\
    C5 & 147 & 427 & 88 & 110 & 51\\
    C6 & 135 & 528 & 106 & 158 & 57 \\
    C.av & 133.00 &436.67 & 78.67 & 117.67 &45.33\\
    N & 69 & 216 & 29 & 59 & 28 \\
    \hline
    C.av/N & 1.93 & 2.02 & 2.71 & 1.99 & 1.62 \\
    \hline
    \end{tabular}
\end{table}
    


  \subsection{ツール使用者の感想}
ここでは最後に,ツールを用いて分析を行った選手による,操作終了後の感想やツール使用中の分析者の呟きをまとめたものを記載する.\ref{table:高校生のツール使用者の感想}
入力する数値が感覚的に分かる値であるため,作業そのものの難易度については簡単だったという意見が多い.一方で入力画面の数値記入場所については
確認画面がないことや数値記入の際に慎重にならざるを得ないという声が見受けられた.
そのため入力する数値よりも,数値入力画面のユーザーインターフェイスの改善が必要であると考えた.


\begin{table}[h]
  \caption{Cチーム選手によるツール使用後の感想}\label{table:高校生のツール使用者の感想}\centering
  \centering
  \begin{tabular}{cll}
    \hline \hline
    被験者 & 感想 \\
    \hline \hline
    C1 & 数字しか打たなかったので、簡単だった。少しだけ入力の縦棒からずれてしまうところ \\
    C2 & が打ちずらかった。保存したかどうかを判断する白丸がわかりやすくて安心した。\\
    & シュートの時に34p→4lとならずに一発で34lと書けたらもう少し楽だった気がする。 \\
    C3 & 数字とローマ字を入力するだけだったので簡単だった。 \\
    C4 & 打った数字があってるかを線で確認できて簡単に作業できた。 \\
    C5 & 12pなどのパスの出し手と受け手をしっかりかかなければいけなかったのが、\\
    & 受け手だけで書けたらもうすこし早くできたきがする。\\
    C6 & 数字を打ち込むだけなので簡単だった。 \\
    C7 & 数字を打ち込むだけで簡単な作業だった。文字が多くて少し目が疲れました。\\
    & 慣れたらとても楽だなと思いました。 \\
    \hline
  \end{tabular}
\end{table}



\end{document}
