\documentclass[../main.tex]{subfiles}
\begin{document}
\chapter{問題}
\par 前章で述べた様に,国際大会や学生選手権における退水セットの重要性は増しているがその分析は主要大会の試合後にとどまり,
日頃の練習から客観的な分析を行うことはできていない.
\par 本研究では,tkinterを用いてボールの軌道を描画することで簡易的な操作ができる上,選手のプレーに影響を与えることのできる,有意性のある分析が出来るという仮説を立てた.
その検証のために退水セットにおける客観的な分析手法を提案するとともに,あらゆる水球関係者が簡易的に分析を行いやすいツールを作成する.
本研究で提案した手法での客観的な分析によるプレーへの影響を評価するために,選手に利用してもらうことで統計情報とアンケートによる選手の使用感を評価した.


\section{本研究で解決する問題}
\par 本研究で作成するツールの目標は「誰もが簡易的に使用でき,かつ選手の動きに影響を与える客観的な分析結果を出力できるツールの作成」である.



\section{課題解決の要件}
\par プレーに影響を与える客観的な出力結果が得られ,かつ誰でも簡単に使用できるツールを作成する要件として,下記項目を定義する.

\subsection{ツール分析前後での統計情報上の変化}
\par 分析ツールの使用前後チームとしてどのような変化があったのか,パス数やシュート数,シュートポジションなどの統計情報をとり,評価する.

\subsection{ツール分析前後での選手の心情の変化}
\par 分析ツールの使用前後でチームとしてどのような変化があった感覚を持つのか,選手へのアンケートから評価を行う.
アンケートはy/n形式によるものと記述式の2つを採用した.

\subsection{分析結果の有用性}
\par 有用性のある分析結果と成ったのかを測定するため,ツール使用後の選手にアンケートをとり,出力結果の有用性を評価する.アンケートは
y/n投票によるものと記述式によるものを採用し,2側面からの評価を行う.

\subsection{出力結果の正確性}
\par 正確な分析を行うため,各選手の出力結果の正確性を評価する.全使用者の出力結果を比較し,使用者が間違いを起こしやすいパターンを計測する.

\subsection{ツール使用時間の短縮}
\par 誰でも,簡単に使うことのできるツール作成のために各選手がツールを使用した時間を計測し,そのばらつきを評価する.特に,ツール作成者との使用時間の差異を
評価することで使用者が扱いにくい作業を特定し,ツール改良に役立てる.

\subsection{ツール使用者の感想}
\par ツールを使用後の感覚について,使用者にアンケートを行い使用感を評価する.アンケートではy/nの投票形式によるものと,記述式の2つを実施する.




\end{document}
