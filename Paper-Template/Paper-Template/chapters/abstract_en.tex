\documentclass[../main.tex]{subfiles}
\begin{document}

Abstract
\begin{center}
    \begin{large}
        \begin{tabular}{cc}
            \hline
                \titleen \\
            \hline
        \end{tabular}
    \end{large}
\end{center}


\par Water polo is a ball game played in a pool 30 meters long and 25 meters wide, 
and the duration of the game is 4 periods of 8 minutes.
Each team has seven players on the field, including one goalkeeper, and the rule is that each offense must 
be played within 30 seconds.Since the rule changes by FINA in 2019, 
referees in water polo are increasingly making the decision 
to exclusion or penalty foul from the water in order to increase scoring opportunities, The importance of the water exclusion set 
is increasing in both offensive and defensive aspects of the game.
In the final round of the 2021 Tokyo Olympics water polo tournament, all teams with a lower percentage of water 
exclusion sets than their opponents were eliminated from the tournament, indicating the importance of water exclusion sets.
In the current Keio University Swimming Club water polo team, the water set is practiced only by analyzing 
statistical information such as the number of shots and pass misses, and by watching a video of the set.
The current Keio University Swimming Team's water polo team only analyzes statistical information such as the number of 
shots and pass misses, as well as the video footage of the practice, which is a subjective analysis.
\par Therefore, in this study, we developed a tool to visualize the ball rolling of the exclusion set using tkinter, 
a standard library included in Python.
In this study, I aimed to investigate the effectiveness 
of using a tool for analyzing statistics and video footage in practices of the exclusion set in water polo. 
I measured changes in the performance of the offensive (OF) and defensive (DF) players in two exclusion set practices, 
comparing a team that used the analysis tool with a team that did not.
 The results showed that in the DF adaptation experiment, the team that did not use the tool repeated 
 the same mistakes in defense, while the team that used the tool was able to recognize and counter the opponent's attacks, 
 preventing the repetition of the same mistakes. Surveys also revealed psychological changes in the players as a result of 
 the objective analysis. However, in the OF adaptation experiment, there was no statistical difference between the team 
 that used the tool and the team that did not. While surveys showed psychological changes, it did not affect the play.

Key Words:\\
\underline{1. Graduation Paper},
\underline{2. Waterpolo},
\underline{3. Exclusion Set},
\underline{4. Analysis}

\rightline{\departmenten}
\rightline{\nameen}

\clearpage

\end{document}
